\documentclass{article}
\usepackage[fleqn]{amsmath}
\usepackage{latexsym}
\usepackage{amsfonts}
\usepackage{amssymb}
\usepackage{graphicx}
\usepackage{epsfig}
\usepackage[ruled,vlined]{algorithm2e}

\begin{document}


\section{Simulation studies}
\subsection{Experimental setup}
We evaluate the performance of our algorithm via simulation in this section.
Our simulation setting is as follows. We randomly create $n$(ranged from $10$ to $100$) nodes
on a $100*100$ $m^2$ square. The data sinks are also randomly located within this square. The maximum rate for each sensor is $100 Kbps$. The maximum transmission range of each sensor is $25 m$. For power consumption model, we set $e_{Rx} = 0.00002$ $J/b$, $e_{Tx}$ $=$ $0.00003$ $nJ/b$ \cite{Wire} Since nodes in our protocol do not 
consume more energy than they can collect, the rate nodes are directly related to the amount of energy that they collect. Take solar energy as an example \cite{Ste}. The total energy collected from a $37 * 33 mm^2$ solar cell over a 48-h period is $655.15 mWh$ for the sunny day and $313.70 mWh$ for the partly cloudy day. Thus we generate $B_V(t+1)$ for every sensor node randomly between $8.19 J$ and $3.92 J$ with every $10 mins$ for a timeslot. We run each experiment over 20 different random topologies. For example when evaluating the impact of $\epsilon$, we create 20 different topologies with both nodes number and location at random, then run algorithms on each of them.\\
We compare the performance and complexity of our algorithm(\textbf{TPath}) with the Garg and K\"{o}nemann's framework algorithm(\textbf{Garg and K\"{o}nemann}).




\bibliographystyle{plain}
\bibliography{Maxi.bib}
\bibliography{Wire.bib}
\bibliography{Ste.bib}




\end{document}