\documentclass{article}
\usepackage[fleqn]{amsmath}
\usepackage{latexsym}
\usepackage{amsfonts}
\usepackage{amssymb}
\usepackage{graphicx}
\usepackage{subfigure}



\begin{document}


\section{Simulation Studies}
\subsection{Experimental Setup}
We evaluate the performance of our algorithm via simulation in this section.
Our simulation setting is as follows. We randomly create $n$(ranged from $50$ to $200$) sensor nodes
on a $100*100\,m^2$ square ~\cite{Maxi}. The base station is located at the square lower left corner this. The maximum rate for each sensor is $100\,Kbps$. The maximum transmission range of each sensor is $25\,m$. For power consumption model, we set $e_{Rx} = 0.00576\,mJ/b$, $e_{Tx} = 0.0144\,mJ/b$ ~\cite{Ene}. Since nodes in our protocol do not
consume more energy than they can collect, the rate nodes are directly related to the amount of energy that they collect. Take solar energy as an example ~\cite{Ste} . The total energy collected from a $37 * 33\,mm^2$ solar cell over a 48-h period is $655.15\,mWh$ for the sunny day and $313.70\,mWh$ for the partly cloudy day. Thus we generate $B_V(t+1)$ for every sensor node randomly between $8.19\,J$ and $3.92\,J$ with every $10\,mins$ for a timeslot. We also set each sensor with a $AA$ Battery for initial.\\
We compare the performance and running time of our algorithm(\textbf{TPath}) with the Garg and K\"{o}nemann's framework algorithm(\textbf{Garg and K\"{o}nemann}). We run each experiment over 20 different topologies.

\subsection{Impact of $\epsilon$ on Performance}
\begin{figure}[htbp]
\centering
\includegraphics[scale=0.6]{performance.pdf}
\caption{The total flow solution by both algorithms with different $\epsilon$ ranged from 0.25 to 0.1 in a wireless sensor network consisting of n sensor nodes with same MaxRate and random BudgetEnergy in 100*100 square}
\label{fig:performance}
\end{figure}

First we investigate the impact of $\epsilon$, the parameter of the approximation algorithm, on the network performance. Figure~\ref{fig:performance} depict the performance delivered by both algorithms with different $\epsilon$ when the number of nodes in the network is 50, 150, 200 and 300 respectively. From figure~\ref{fig:performance}, we observe that with the $\epsilon$ fall from 0.3 to 0.1, the performance gap between two algorithms are smaller and smaller. To be specific, our algorithm generate better solution than the Garg and K\"{o}nemann's framework algorithm when nodes number is 200 and 300. The interesting result shows that our algorithm could performance as well as the Garg and K\"{o}nemann's framework algorithm with proper approximation parameter(e.g. $\epsilon = 0.1$) even better.

\subsection{Running Time Analysis}
\begin{figure}[htbp]
\centering
\includegraphics[scale=0.6]{running.pdf}
\caption{The running time of both algorithms with specific $\epsilon$ in different wireless sensor network consisting of n sensor (n ranged form 50 to 150) nodes with same MaxRate and random BudgetEnergy in 100*100 square}
\label{fig:running}
\end{figure}
We then analyse the running time differences between both algorithms with different $\epsilon$ in a test-bed with a processor running at $2.40\,GHz$ and $3.86\,GB$ RAM when the nodes number ranged from 50 to 150.
Figure~\ref{fig:running} shows that the running time growth trend of both algorithms when the sensor nodes increaseing from 50 to 150. Obviously, running time is a key feature especially in large-scale network. The result shows that the Garg and K\"{o}nemann's framework algorithm
grows much faster than our algorithm. Our explanation for such a phenomenon is as follows. The Garg and K\"{o}nemann's framework algorithm compute shortest path per node per loop, while our algorithm only do shortest path computing one time per loop through the $DS(e)$ assistance.












\bibliographystyle{plain}
\bibliography{Maxi,Ste,Ene}


\end{document} 